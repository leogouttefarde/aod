\documentclass[a4paper, 10pt, french]{article}
% Préambule; packages qui peuvent être utiles
   \RequirePackage[T1]{fontenc}        % Ce package pourrit les pdf...
   \RequirePackage{babel,indentfirst}  % Pour les césures correctes,
                                       % et pour indenter au début de chaque paragraphe
   \RequirePackage[utf8]{inputenc}   % Pour pouvoir utiliser directement les accents
                                     % et autres caractères français
   \RequirePackage{lmodern,tgpagella} % Police de caractères
   \textwidth 17cm \textheight 25cm \oddsidemargin -0.24cm % Définition taille de la page
   \evensidemargin -1.24cm \topskip 0cm \headheight -1.5cm % Définition des marges
   \RequirePackage{latexsym}                  % Symboles
   \RequirePackage{amsmath}                   % Symboles mathématiques
   \RequirePackage{tikz}   % Pour faire des schémas
   \RequirePackage{graphicx} % Pour inclure des images
   \RequirePackage{listings} % pour mettre des listings
% Fin Préambule; package qui peuvent être utiles

\title{Rapport de TP 4MMAOD : Génération de patch optimal\\
``{\em Question 2 : Modélisation par Bellman}'' }
\author{
GOUTTEFARDE Léo (groupe 5)
\\ PIELLARD Jérémie (groupe 2)
}

\begin{document}

\maketitle

%%%%%%%%%%%%%%%%%%%%%%%%%%%%%%%%%%%%%%%%%%%%%%
\section*{Modélisation du problème général par équation de Bellman}

\subsection*{Données}

\begin{itemize}
\item On cherche à créer un patch $P$ minimal qui transforme un fichier $F_1$ de $n$ lignes en un fichier $F_2$ de $m$ lignes.

\item On note $F_1(i)$, la ligne $i \in 1 \cdots n$ du fichier 1 et $F_2(j)$ la ligne $j \in 1 \cdots m$ du fichier 2.

\item On définit $L(F_i(j))$ comme le nombre de caractères de la ligne $F_i(j)$ (incluant le caractère de fin de ligne).


\item Soit $(i,j)$ l'état dans lequel on a utilisé les $i$ premières lignes du fichier $F_1$ de départ pour générer les $j$ premières lignes du fichier de sortie.

\item Soit $C^*_{ij}$ le coût optimal du patch transformant les $i$ premières lignes du fichier $F_1$ d'entrée en les $j$ premières lignes du fichier de sortie.
\end{itemize}
\vspace{3mm}


\subsection*{Variables}
On introduit les variables suivantes :
\\

\begin{itemize}
\item[$\bullet$]
$u_{ij}$ : décision à l'état $(i,j)$
\\
\item[$\bullet$]
$h_{ij}$ : coût de substitution de la ligne $F_1(i)$ par la ligne $F_2(j)$.\\
Vaut 0 si $F_1(i) = F_2(j)$, $10 + L(F_2(j))$ sinon
\end{itemize}
\vspace{6mm}


On introduit les variables binaires (de valeur 1 ou 0) suivantes :
\\

\begin{itemize}
\item[$\bullet$]
$s_{ij}$ : modélise la substitution (ou non) de la ligne $F_1(i)$ par la ligne $F_2(j)$
\\
\item[$\bullet$]
$d_i$ : modélise la destruction (ou non) de la ligne $F_1(i)$
\\
\item[$\bullet$]
$D_i^w$ : modélise la destruction (ou non) des $m$ lignes de la ligne $k$ à la ligne $k+m-1$ de $F_1$
\\
\item[$\bullet$]
$a_j$ : modélise l'ajout (ou non) de la ligne $F_2(j)$
\end{itemize}
\vspace{3mm}


\subsection*{Objectif}

Pour tout état $(i,j)$, il y a 4 opérations possibles :
\begin{enumerate}
\item Ajout [+] : aller vers l'état $(i,j+1)$, en ajoutant la ligne $B_j$ au fichier $F_1$ d'entrée. Coût = $10 + L_j^B$

\item Substitution [=] : aller vers l'état $(i+1,j+1)$, en substituant la ligne $F_1(i)$ par la ligne $F_2(j)$. Coût = 0 si $F_1(i) = F_2(j)$, 10 + $L_j^B$ sinon.

\item Destruction simple [d] : aller vers l'état $(i+1,j)$ en détruisant la ligne $i$ du fichier d'entrée. Coût = 10

\item Destruction multiple [D] : aller vers l'état $(i+w,j),\ w \in 2 \cdots n$ en détruisant les lignes $i \cdots i + w-1$ du fichier d'entrée. Coût = 15
\end{enumerate}
\vspace{3mm}

Ainsi on obtient l'équation de Bellman suivante :

$$\begin{array}{rll}
C^*_{ij}(a_{j},s_{ij},d_{i},D_{i}^w, u_{ij}) = \min ( & 10 + C^*_{i+1,j}(a_{j},s_{i+1,j},d_{i+1},D_{i+1}^w, u_{ij}),\\
& 15 + C^*_{i+k,j}(a_{j},s_{i+k,j},d_{i+k},D_{i+k}^w, u_{i+k,j}),\\
& h_{ij} + C^*_{i+1,j+1}(a_{j+1},s_{i+1,j+1},d_{i+1},D_{i+1}^w, u_{i+1,j+1}),\\
& 10 + L(F_2(j)) + C^*_{i,j+1}(a_{j+1},s_{i,j+1},d_{i},D_{i}^w, u_{i,j+1}) \hspace{4mm} )
\end{array}$$

% \noindent 



\end{document}
%% Fin mise au format


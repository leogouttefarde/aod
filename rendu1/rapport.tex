\documentclass[a4paper, 10pt, french]{article}
% Préambule; packages qui peuvent être utiles
   \RequirePackage[T1]{fontenc}        % Ce package pourrit les pdf...
   \RequirePackage{babel,indentfirst}  % Pour les césures correctes,
                                       % et pour indenter au début de chaque paragraphe
   \RequirePackage[utf8]{inputenc}   % Pour pouvoir utiliser directement les accents
                                     % et autres caractères français
   \RequirePackage{lmodern,tgpagella} % Police de caractères
   \textwidth 17cm \textheight 25cm \oddsidemargin -0.24cm % Définition taille de la page
   \evensidemargin -1.24cm \topskip 0cm \headheight -1.5cm % Définition des marges
   \RequirePackage{latexsym}                  % Symboles
   \RequirePackage{amsmath}                   % Symboles mathématiques
   \RequirePackage{tikz}   % Pour faire des schémas
   \RequirePackage{graphicx} % Pour inclure des images
   \RequirePackage{listings} % pour mettre des listings
% Fin Préambule; package qui peuvent être utiles

\title{Rapport de TP 4MMAOD : Génération de patch optimal\\
``{\em Question 1 : Modélisation par PLNE}'' }
\author{
GOUTTEFARDE Léo (groupe 5)
\\ PIELLARD Jérémie (groupe 2)
}

\begin{document}

\maketitle

%%%%%%%%%%%%%%%%%%%%%%%%%%%%%%%%%%%%%%%%%%%%%%
\section*{Modélisation du problème restreint sous forme de PLNE}

\subsection*{Données}
\noindent On cherche à créer un patch $P$ restreint minimal qui transforme un fichier $F_1$ de $n$ lignes en un fichier $F_2$ de $m$ lignes.

\noindent Pour toute ligne $i$ de $F_1$, notée $F_1(i)$, on a donc $i \in 1 \cdots n$.

\noindent De même pour toute ligne $j$ de $F_2$, notée $F_2(j)$, on a donc $j \in 1 \cdots m$.

\noindent Finalement, on définit $L(F_2(j))$ comme le nombre de caractères de la ligne $F_2(j)$ (incluant le caractère de fin de ligne).

\subsection*{Variables}
On introduit les variables de coût suivantes :
\\

\begin{itemize}
\item[$\bullet$]
$c_s(i, j)$ : modélise le coût de substitution de la ligne $F_1(i)$ par la ligne $F_2(j)$
\\
\item[$\bullet$]
$c_a(i, j) = 10 + L(F_2(j))$ : modélise le coût d'ajout de la ligne $F_2(j)$ après la ligne $F_1(i)$ 
\\
\item[$\bullet$]
$c_d(i) = 10$ : modélise le coût de destruction de la ligne $F_1(i)$ (identique pour tout $i$, peu utile donc)
\end{itemize}
$\newline$
\\
\indent On introduit les variables binaires (de valeur 1 ou 0) suivantes :
\\

\begin{itemize}
\item[$\bullet$]
$s(i, j)$ : modélise la substitution (ou non) de la ligne $F_1(i)$ par la ligne $F_2(j)$
\\
\item[$\bullet$]
$d(i)$ : modélise la destruction (ou non) de la ligne $F_1(i)$
\\
\item[$\bullet$]
$a(i, j)$ : modélise l'ajout (ou non) de la ligne $F_2(j)$ après la ligne $F_1(i)$
\end{itemize}

% \begin{equation}
% c_s(i, j) = 10 + L(F_2(j))
% \end{equation}
\subsection*{Contraintes}
% \paragraph{Justification\,: }
% {
%   loltest
% }
Les contraintes de coût suivantes sont à considérer pour $c_s$ :
\\
\begin{itemize}
\item[$\bullet$]
$c_s(i, j) = 0$ si $F_1(i) = F_2(j)$ : le coût de copie des lignes laissées telles quelles est nul
\\
\item[$\bullet$]
$c_s(i, j) = 10 + L(F_2(j))$ si $F_1(i) \ne F_2(j)$ : coût des substitutions classiques
\end{itemize}
$\newline$
\\
\indent Pour l'ajout et la destruction, on peut d'abord distinguer plusieurs cas :

\subsubsection*{Cas 1 : Nombre minimal de destructions $(n > m)$}

Si $n > m$, on aura minimum $n-m$ destructions à effectuer, soit un coût minimal de $10 \, (n - m)$.


\subsubsection*{Cas 2 : Nombre minimal d'ajouts $(n < m)$}

Si $n < m$, on aura minimum $m - n$ ajouts à effectuer, soit un coût minimal de

$10 \, (m - n) + \sum\limits_{j=0}^m a(j) \cdot L(F_2(j))$.


\subsubsection*{Cas 3 : Même nombre de lignes dans chaque fichier $(n = m)$}

Si $n = m$, il n'y a pas forcément d'ajout ni de destruction de ligne.
\\

\noindent Finalement, il ne peut pas y avoir à la fois une substitution et une destruction sur une même ligne $i$ de $F_1$.
\\

\subsection*{Objectif à minimiser}

On obtient la fonction objectif suivante :
\begin{equation*}
\min_{s, d, a} \, \sum\limits_{i,j} c_s(i, j) \cdot s(i, j) + \sum\limits_{i,j} c_a(i, j) \cdot a(i, j) + 10 \sum\limits_i d(i, j)
\end{equation*}




%%%%%%%%%%%%%%%%%%%%%%%%%%%%%%%%%%%%%%%%%%%%%%
% \section{Compte rendu d'expérimentation (2 points)}
%   \subsection{Conditions expérimentaless}
%      {lolsubsec
%      }

%     \subsubsection{Description synthétique de la machine\,:} 
%       {lolsubsub
%       }

\end{document}
%% Fin mise au format

